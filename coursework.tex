\documentclass[a4paper,14pt,russian]{article}
\usepackage{./coursework}
\usepackage{lastpage}
\newcommand{\caphedre}{ВТ}
\newcommand{\theme}{Проектирование локальной вычислительной сети}
\newcommand{\groupnumber}{5305}
\newcommand{\studentname}{Билькис П.П.}
\newcommand{\teachername}{Белова Е.Ю.}
\newcommand{\discipline}{Сети ЭВМ}
\newcommand{\myyear}{2018}
\begin{document}

\input{./coursework-titul.tex}

\section*{Задание на курсовой проект}

Студент \studentname\\
Группа \groupnumber\\\\
Тема работы: \theme\\
Исходные данные:\\
Профиль организации, количество зданий, количество этажей в каждом здании, минимальное количество сотрудников, минимальное количество серверов.
Содержание пояснительной записки:\\
«Аннотация», «Содержание», «Введение», «Описание организации», «Разработка транспортной подсистемы локальной вычислительной сети», «Расчет стоимости локальной вычислительной сети», «Список использованных источников», «Заключение», «Приложения».
Предполагаемый объем пояснительной записки:\\
Не менее \pageref{LastPage}
Дата выдачи задания: 01.02.2018\\
Дата сдачи реферата: 17.05.2018\\
Дата защиты реферата: 17.05.2018\\

\vfill


Студент группы \groupnumber \hfill \studentname


Преподаватель \hfill \teachername


\clearpage

\selectlanguage{russian}
\begin{abstract}
  В курсовом проекте необходимо выполнить проектирование локальной вычислительной сети производства одежды. В первой части проекта приведено описание функций и структуры производства, направлений работы его подразделений. Вторая часть проекта включает в себя разработку транспортной подсистемы локальной вычислительной сети производства одежды. В третьей части проекта выполнена оценка стоимости локальной вычислительной сети и её дальнейшая эксплуатация, выдвинуты требования к составу обслуживающего персонала.
\end{abstract}


\selectlanguage{english}
\begin{abstract}
  In this coursework local computer network for the clothes manufacture is described. The first section presented basic stucture \& functions of the manufacture, it's subdivisions and their responsibilities are depicted. The second, however, describes transport sub-system of the local computer network. The third section contains of cost estimations for bulding and supporting this local computer network, and staff requirements.
\end{abstract}

\clearpage

\selectlanguage{russian}
\tableofcontents
\clearpage

\section*{Введение}
\addcontentsline{toc}{section}{Введение}
Данный курсовой проект предусматривает создание локальной вычислительной сети для производства одежды. Осуществляется это с помощью первичного проектирования планировки здания учреждения с последующей установкой необходимых устройств коммутации и их настройки.
\clearpage

\section{Описание организации}

\subsection{Функции организации}
% Функции организации
В рамках курсового проекта разрабатывается корпоративная сеть для производства одежды. Продукция производства - мужская и женская одежда, широкий спектр: от нижнего белья до верхней одежды из кожи.
Производство представляет из себя Закрытое Акционерное Общество (ЗАО).1

\subsection{Структура организации и функции подразделений}
% Структура организации и функции подразделений
Предприятие состоит из Административной части, Основного производства и Вспомогательного производства.

\subsubsection{Административная часть}
Административная часть состоит из следующих подразделений:
\begin{description}
\item [Отдел кадров]
  Отдел осуществляет контроль за текущей потребностью в кадрах, составляет списки резерва кадров, учувствует в работе по отбору кандидатов на работу, ведёт учет личного состава, оформляет документы о приеме сотрудников на работу, ведёт установленную документацию по кадрам.
\item [Отдел закупок]
  Отдел осуществляет поиск, анализ данных, выбор поставщиков необходимого оборудования. Проверяет поступающую продукцию.
\item [Отдел продаж]
  Отдел осуществляет рекламную деятельность, направленную на увеличение продаж продукции. Производит работу с клиентами, направленную на поддержание связей и увеличения объёмов взаимного сотрудничества.
\item [Отдел технического обеспечения]
  Отдел осуществляет ремонт или заменую сломанного оборудования.
\item [Бухгалтерия]
  Бухгалтерия занимается ведением бухгалтерского, налогового и управленческого учета финансово-хозяйственной деятельности завода, формирует бухгалтерскую, налоговую и управленческую отчетность, взаимодействует с государственными налоговыми органами, взаимодействует с финансовыми организациями в пределах своей компетенции, осуществляет платежи.
\item [IT-отдел]
  Отдел обеспечивает работоспособность локальной компьютерной сети, серверов и рабочих станций пользователей, обеспечивает информационную безопасность, антивирусную защиту, проводит техническое обслуживание и организацию ремонта вычислительной и оргтехники, обеспечивает рабочие станции и сервера свободным программным обеспечением.
\item [Хозяйственный отдел]
  Отдел обеспечивает хозяйственное, материально-техническое и социально-бытовое обслуживание издательства, контролирует исправность оборудования (лифтов, освещения, систем отопления и др.), оформляет документы, необходимые для заключения договоров на приобретение оборудования, обеспечивает подразделения канцелярскими принадлежностями, оборудованием, оргтехникой, мебелью, хозяйственными товарами, обеспечивает сохранность вышеуказанного оборудования, оформляет документы на техническое обслуживание и ремонт оргтехники и оборудования, занимается содержанием в надлежащем состоянии зданий и помещений предприятия.
\end{description}

\subsubsection{Основное производство}
\begin{description}
\item [Цех по пошиву лёгкого женского платья]
  Цех занимается пошивом лёгкого женского платья из готовых раскроек.
\item [Цех по пошиву головных уборов]
  Цех занимается пошивом головных уборов из готовых раскроек.
\item [Цех по пошиву женской верхней одежды]
  Цех занимается пошивом женской верхней одежды из готовых раскроек.
\item [Цех по пошиву мужской верхней одежды]
  Цех занимается пошивом мужской верхней одежды из готовых раскроек.
\item [Цех по пошиву изделий из кожи]
  Цех занимается пошивом и раскройкой изделий из кожи.
\item [Раскройный цех]
  Цех готовит раскройки для пошива изделий во всех пошивных цехах (кроме цеха по пошиву изделий из кожи).
\item [Экспериментальный цех]
  В экспериментальном цеху проходят апробацию новейшие технологии, методы, оборудования и приёмы для изготовления одежды.
\end{description}
\subsubsection{Вспомогательное производство}
\begin{description}
\item [Ремонтно-механический цех]
  Ремонтно-механический цех занимается текущим ремонтом и обслуживанием оборудования производства.
\end{description}

\section{Разработка транспортной подсистемы локальной вычислительной сети}

\subsection{Описание поэтажных планов зданий организации}
% Описание поэтажных планов зданий организации

\subsection{Выбор топологии локальной вычислительной сети}
% Выбор топологии локальной вычислительной сети

\subsection{Соединение удалённого коммуникационного оборудования}
% Соединение удалённого коммуникационного оборудования
Для соединения коммутационного оборудования, которое находятся в различных зданиях, кабель витая пара чаще всего не используется, поэтому целесообразно рассмотреть две технологии: VPN и волокно-оптические системы связи.
VPN (Virtual Private Networks – Виртуальные частные сети) технология используется при объединении нескольких сетей, в одну виртуальную сеть, позволяя пользователям использовать ресурсы сетей всех офисов компании, как единую сеть. Отличное VPN сети от локальной сети состоит в том, что в VPN для передачи данных используется опорные сети оператора и Интернет каналы, а не кабель витой пары. Вся передаваемая по VPN информация надежно шифруется. 
Волокно-оптическая система связи (ВОЛС) – это вид системы передачи, при котором информация передается по оптическим диэлектрическим волноводам, известным под названием оптическое волокно. ВОЛС используется при построении объектов, в которых СКС должна объединить многоэтажное здание или при объединении территориально-разрозненных зданий. Оптически кабель может быть многомодовым (обеспечивает передачу сигналов на расстояние 1-5 км) и одномодовым (обеспечивает передачу сигналов на расстояние в десятки километров). Скорость передачи данных по оптоволокну ограничивается только пропускной способностью передающего и приемного модуля системы. Как правило, передача данных по оптоволокну составляет от 1 до 10 Гбит/с. Поэтому оптоволоконные системы чаще всего используется для передачи больших объемов информации, в том числе аудио и видеосигналов.
В рамках курсового проекта выбрана ВОЛС, поскольку её целесообразно использовать для соединения коммутационного оборудования, которое расположено в пределах километра друг от друга. Для прокладки ВОЛС применяется подземный метод, суть которого заключается в осуществлении монтажа кабеля по подземным коммуникациям. Данный метод позволяет оградить ВОЛС от нежелательных повреждений и негативного воздействия окружающей среды.
Для соединения двух удаленных маршрутизаторов друг с другом, необходимо наличие модулей SFP. Данные модули используются для присоединения платы маршрутизатора к ВОЛС.

\subsection{Выбор серверного оборудования}
% Выбор серверного оборудования
К коммутатору 3 уровня, который находится в главной серверной (монтажный шкаф 3) будут подключены следующие сервера:
\begin{description}
\item[Файловый сервер]
  Данный сервер представляет собой компьютер, первичной целью которого является обеспечение доступа к файлам (таких как документы, звуковые файлы, фотографии, изображения и т.д.), размещенных на его устройствах хранения информации, другим компьютерам издательства. Плюсами использования сервера являются: экономия пространства жесткого диска персональных компьютеров, обеспечение совместной работы пользователей с информационными ресурсами, надежность хранения информации. Сервер в локальной сети использует протокол SMB/CIFS (Windows и Unix-подобные операционные системы). Клиенты соединяются с сервером, используя протоколы TCP. После того, как соединение установлено, клиенты могут посылать команды серверу (эти команды называются SMB-команды), который дает им доступ к ресурсам, позволяет открывать, читать файлы, писать в файлы и, вообще, выполнять весь перечень действий, которые можно выполнять с файловой системой. В случае SMB, данные действия совершаются через сеть.

\item[Сервер печати]
  Он обеспечивает совместный доступ к принтеру, подключённому к серверу для определённого списка машин локальной сети.
\item[Сервер электронной почты]
  Сервер обеспечивает:
  \begin{itemize}
  \item Прием электронной почты завода из Интернет и ее раскладку по отдельным ящикам внутри общего домена
  \item Проверку входящей почты на наличие вирусов и их обезвреживание в случае необходимости
  \end{itemize}
  Размеры почтовых ящиков пользователей ограничены только возможностями сервера. На сервере организован электронный документооборот и обмен сообщениями внутри издательства. Вся конфиденциальная информация остается в пределах локальной сети.
  Когда пользователь набрал сообщение и посылает его получателю, почтовый клиент взаимодействует с почтовым сервером, используя протокол SMTP. Почтовый сервер отправителя взаимодействует с почтовым сервером получателя. На почтовом сервере получателя сообщение попадает в почтовый ящик, откуда при помощи агента доставки сообщений (MDA) доставляется клиенту получателя. Для финальной доставки полученных сообщений используется протокол POP3 или IMAP.
\item[Сервер баз данных]
  Сервер БД обслуживает базу данных и отвечает за целостность и сохранность данных, а также обеспечивает операции ввода-вывода при доступе к информации. Применяемая база данных – MySQL. В базе данных содержится полная информация о перечне часов, выпускаемых заводом.
\item[Веб-сервер]
  Данный сервер обеспечивает размещение и выдачу по запросу любой информации в формате HTML, что позволяет издательству организовать собственное Интернет представительство, прорекламировать свои услуги и продукты, довести до конечных пользователей прейскурант, получить обратную связь.
  Клиент, которым обычно является веб-браузер, передаёт веб-серверу запросы на получение ресурсов, обозначенных URL-адресами. Ресурсы – это HTML-страницы, изображения, файлы, медиа-потоки или другие данные, которые необходимы клиенту. В ответ веб-сервер передаёт клиенту запрошенные данные. Такой обмен происходит по протоколу HTTP.
\item[Сервер резервного копирования файлов]
  Сервер производит копирование наиболее ценной информации на магнитную ленту или магнитооптические диски. Если какой-либо из серверов полностью вышел из строя или вся база данных была случайно удалена, сервер резервного копирования становиться незаменимым. Сервер резервного копирования может некорректно сохранить некоторые данные (например, сообщения с почтового сервера). Чтобы этого не происходило, существуют программы-агенты для каждого из сетевых приложений. Они поставляются как дополнение к серверу резервного копирования.
\item[Прокси-сервер]
  Данный сервер действует как посредник, помогая пользователям получить информацию из Интернета (трансляция запросов к WWW и FTP ресурсам) и при этом, обеспечивая защиту сети, может сохранять часто запрашиваемую информацию в кэш-памяти на локальном диске, быстро доставляя её пользователям без повторного обращения к Интернету, делая потребление Интернет трафика более экономичным. Кроме того, он может вводить ограничение доступа к ресурсам Интернет для рабочих станций локальной сети в соответствии с внутрикорпоративной политикой безопасности.
  Прокси-сервер сконфигурирован таким образом, что принимает или отвергает определённые типы сетевых запросов, поступающие как из локальной сети, так и из Интернета. В такой конфигурации прокси-сервер становиться межсетевым экраном – брандмауэром. Он обеспечивает:
  \begin{itemize}
  \item Защиту локальной сети предприятия от несанкционированного и неправомерного доступа из сети Интернет к ресурсам локальной сети организации;
  \item Трансляцию внутренних адресов локальной сети в публичные Интернет адреса (NAT – network address translation);
  \item Ограничение доступа в Интернет машин локальной сети (Proxy, ACL);
  \item Ограничение ресурсов сети Интернет для доступа из локальной сети (возможно разграничение доступа к серверам WWW в зависимости от содержания).
  \end{itemize}
\end{description}

\subsection{Адресация}
% Адресация

Протокол IP (Internet Protocol) входит в состав стека протоколов TCP/IP и является основным протоколом сетевого уровня, использующимся в Интернет и обеспечивающим единую схему логической адресации устройств в сети и маршрутизацию данных. Подтверждение получения пакетов и повторное обращение за потерянными пакетами входят в круг обязанностей протокола более высокого уровня, например TCP.
С точки зрения протокола IP, сеть рассматривается как логическая совокупность взаимосвязанных объектов, каждый из которых представлен уникальным IP-адресом, называемых узлами (IP-узлами) или хостами (host).
IP-адрес – это уникальный числовой адрес, однозначно идентифицирующий узел, группу узлов или сеть. IP-адрес имеет длину 4 байта и обычно записывается в виде четырех чисел (так называемых “октетов”), разделенных точками – W.X.Y.Z , каждое из которых может принимать значения в диапазоне от 0 до 255.
Существует 5 классов IP-адресов – A, B, C, D, E. Принадлежность IP-адреса к тому или иному классу определяется значением первого октета (W). В таблице \ref{table:ipexample} показано соответствие значений первого октета и классов адресов.

\begin{table}[h]
  \centering
  \begin{tabular}{l|l|l|l|l|l}
    Класс IP-адресса & A & B & C & D & E \\ \hline
    Диапазон первого октета & 1-127 & 128-191  & 192-223 &224-239  & 240-247 \\ \hline
  \end{tabular}
  \caption{Диапазон первого октета для различных классов IP-адресов}
  \label{table:ipexample}
\end{table}

Частное адресное пространство определяется следующими адресными блоками:
\begin{itemize}
\item От 10.0.0.1 до 10.255.244.254
\item От 172.16.0.1 до 172.31.255.254
\item От 192.168.0.1 до 192.168.255.254
\end{itemize}
Данные адреса используются в локальных сетях небольших организаций и не требуют регистрации.
Если количество компьютеров в сети не будет превышать 254, то рекомендуется использовать адреса из диапазона от 192.168.0.1 до 192.168.0.254 с маской подсети 255.255.255.0. Тогда 192.168.0.0 будет номер сети, а адреса компьютеров от 1 до 254. Если компьютеров будет больше, чем 254, то можно использовать диапазон от 172.16.0.1 до 172.31.255.254 с маской подсети 255.255.0.0. Тогда 172.16.0.0 будет номером сети, а адреса компьютеров от 0.1 до 255.254 (это более 65000 адресов).
Присваивается IP-адрес компьютеру либо вручную (статический адрес), либо компьютер получает его автоматически с сервера (динамический адрес). Статический адрес прописывается администратором сети в настройках протокола TCP/IP на каждом компьютере сети и жестко закрепляется за компьютером. В курсовом проекте используется статическая адресация.
IP-маршрутизация – процесс выбора пути для передачи пакета в сети. Под путем (маршрутом) понимается последовательность маршрутизаторов, через которые проходит пакет по пути к узлу-назначению. IP-маршрутизатор – это специальное устройство, предназначенное для объединения сетей и обеспечивающее определение пути прохождения пакетов в составной сети. Маршрутизатор должен иметь несколько IP-адресов с номера сетей, соответствующими номерам объединяемых сетей. Маршрутизация осуществляется на узле-отправителе в момент отправки IP-пакета, а затем на IP-маршрутизаторах.
Когда требуется отправить пакет узлу с определенным IР-адресом, то узел-отправитель выделяет с помощью маски подсети из собственного IР-адреса и IР—адреса получателя номера сетей. Далее номера сетей сравниваются и если они совпадают, то пакет направляется непосредственно получателю, в противном случае - маршрутизатору, чей адрес указан в настройках протокола IP. Если ни узле не настроен адрес маршрутизатора, то доставка данных получателю, расположенному в другой сети, окажется невозможной.
Выбор пути на маршрутизаторе осуществляется на основе информации, представленной в таблице маршрутизации. Таблица маршрутизации – это специальная таблица, сопоставляющая IP-адресам сетей адреса следующих маршрутизаторов, на которые следует отправлять пакеты с целью их доставки в эти сети. Обязательной записью в таблице маршрутизации является так называемый маршрут по умолчанию, содержащий информацию о том, как направлять пакеты в сети, адреса которых не присутствуют в таблице, поэтому нет необходимости описывать в таблице маршруты для всех сетей. Таблицы маршрутизации могут строится «вручную» администратором или динамически с помощью специальных протоколов.

В таблице \ref{table:ipnumtable} приведены сведения о требуемом числе IP-адресов для каждого подразделения производства.



\begin{table}[!htbp]
  \centering
  \begin{tabular}{p{2cm}|p{5cm}|p{2cm}|p{2cm}|p{2cm}|p{2cm}|p{2cm}}
    № здания & Название \newline подразделения & Число \newline рабочих мест & Число IP-адресов & Число IP-адресов в подсети & Число IP-адресов в резерве & Количество портов коммутатора \\ \hline
    \multirow{4}{*}{1} & Отдел кадров & 20+2 & 25 & 32 & 7 & 24  \\ 
             & Отдел закупок & 15+2 & 20 & 32 & 12  & 24 \\ 
             & Отдел продаж & 15+2 & 20 & 32 &  12 & 24 \\ 
             & Бухгалтерия & 20+2 & 25 & 32 & 7  & 24 \\ \hline
    \multirow{3}{*}{2} & Отдел технического обеспечения & 10+1 & 14 & 16 & 2 & 24   \\ 
             & Хозяйственный отдел & 15+2 & 20 & 32 & 12 & 24   \\ 
             & IT-отдел & 5+1 & 9 & 16 & 7 & 24  \\ \hline
    \multirow{2}{*}{3} & Цех по пошиву женской верхней одежды & 5+2 & 10 & 16 & 6 & 24 \\ 
             & Цех по пошиву лёгкого женского платья & 5+2 & 10 & 16 & 6 & 24 \\ \hline
    4 & Цех по пошиву мужской верхней одежды & 5+2 & 10 & 16 & 6 & 24 \\ \hline
    5 & Склад & 10+4 & 17 & 32 & 15 & 24 \\  \hline
    \textbf{Итого} & & 125+22 & 180 & 272 & 92 & 24 (11 штук)\\ \hline
  \end{tabular}
  \caption{Сводная таблица по требуемому количеству IP-адресов}
  \label{table:ipnumtable}
\end{table}


В ЛВС находятся 125 компьютеров. Согласно теоретическим сведениям, если количество компьютеров не превышает 254, то рекомендуется использовать адреса из диапазона от 192.168.0.1 до 192.168.0.254 с маской подсети 255.255.255.0. Тогда 192.168.0.0 будет номер сети, а адреса компьютеров от 1 до 254.
Исходя из таблицы \ref{table:ipnumtable} необходимо 6 подсетей на 32 адреса и 5 подсетей на 16 адресов. Адреса сетей и диапазоны адресов узлов представлены в таблице \ref{table:ipfinal}

\begin{table}[!htbp]
  \centering
  \begin{tabular}{|l|l|l|l|}
    Адрес подсети & Диапазон адресов узлов подсети & Адрес широковещательной  рассылки & Маска подсети \\ \hline
    \multicolumn{4}{c}{6 подсетей по 32 адреса} \\ \hline
    192.168.0.0 & 192.168.0.1-192.168.0.30 & 192.168.0.31 & \multirow{6}{*}{255.255.255.224} \\ \cline{1-3}
    192.168.0.32 & 192.168.0.33-192.168.0.62 & 192.168.0.63  &  \\ \cline{1-3}
    192.168.0.64 & 192.168.0.65-192.168.0.94 & 192.168.0.95  &  \\ \cline{1-3}
    192.168.0.96 & 192.168.0.97-192.168.0.126 & 192.168.0.127 &  \\ \cline{1-3}
    192.168.0.128 & 192.168.0.129-192.168.0.158 & 192.168.0.159 &  \\ \cline{1-3}
    192.168.0.160 & 192.168.0.161-192.168.0.190 & 192.168.0.191 &  \\ \hline
    \multicolumn{4}{c}{5 подсетей по 16 адресов} \\ \hline
    192.168.1.0 & 192.168.1.1-192.168.1.14 & 192.168.1.15 & \multirow{5}{*}{255.255.255.240} \\ \cline{1-3}
    192.168.1.16 & 192.168.1.17-192.168.1.30 & 192.168.1.31 &  \\ \cline{1-3}
    192.168.1.48 & 192.168.1.49-192.168.1.62 & 192.168.1.63 &  \\ \cline{1-3}
    192.168.1.64 & 192.168.1.65-192.168.1.78 & 192.168.1.79 &  \\ \cline{1-3}
    192.168.1.80 & 192.168.1.81-192.168.1.94 & 192.168.1.95 &  \\ \cline{1-3}
    
  \end{tabular}
  \caption{Распределение IP адресов}
  \label{table:ipfinal}
\end{table}



\subsection{Выбор коммутационного оборудования}
% Выбор оборудования
\subsubsection{Количество коммутационного оборудования}
В каждом корпусе располагается монтажный шкаф или стойка, в котором установлено коммутационное оборудование.
Количество портов коммутатора и маршрутизатора зависит от числа компьютеров, которые необходимо подключить к сети.\\
Каждый компьютер соединяется с информационной розеткой посредством патч-корда, длиной 3 метра. Кабельная трасса от информационной розетки до патч-панели, расположенной в монтажном шкафу или стойке, проходит в кабель-канале. Соответствующие порты патч-панели и коммутатора соединяются друг с другом при помощи патч-корда, длиной 0,5 метра. Один порт в коммутаторе зарезервирован для соединения коммутатора с маршрутизатором. Маршрутизаторы соединяются между собой с помощью ВОЛС, поэтому предусмотрено наличие SFP-модуля для каждого маршрутизатора. Количество требуемое коммутационное и сопутствующее оборудование сведено в таблицу \ref{table:hw}

\begin{table}[h]
  \centering
  \begin{tabular}{|p{1cm}|p{1cm}|p{5cm}|p{2cm}|p{2cm}|}
    \hline
   № здания & № стойки & Название оборудования \newline или комплектующих & Количество & Количество \newline портов \\ \hline
    \multirow{4}{*}{1} & \multirow{4}{*}{1} & Коммутатор 3-го уровня & 1 & 8 \\ \cline{3-5}
            & & Коммутатор 2-го уровня & 4 & 24 \\ \cline{3-5}
            & & Патч-панель & 2 & 48 \\ \cline{3-5}
            & & ИБП & 1 & - \\ \hline
    \multirow{5}{*}{2} & \multirow{5}{*}{2} & Коммутатор 3-го уровня & 1 & 8 \\ \cline{3-5}
            & & Коммутатор 2-го уровня & 3 & 24 \\ \cline{3-5}
            & & Патч-панель & 1 &  48 \\ \cline{3-5}
            & & Патч-панель & 1 &  24 \\ \cline{3-5}
            & & ИБП & 1 & - \\ \hline
    \multirow{4}{*}{3} & \multirow{4}{*}{3} & Коммутатор 3-го уровня & 1 & 8 \\ \cline{3-5}
            & & Коммутатор 2-го уровня & 2 & 24 \\ \cline{3-5}
            & & Патч-панель & 1 & 48 \\ \cline{3-5}
            & & ИБП & 1 & - \\ \hline
    \multirow{4}{*}{4} & \multirow{4}{*}{4} & Коммутатор 3-го уровня & 1 & 8 \\ \cline{3-5}
            & & Коммутатор 2-го уровня & 1 & 24 \\ \cline{3-5}
            & & Патч-панель & 1 & 24 \\ \cline{3-5}
            & & ИБП & 1 & - \\ \hline
    \multirow{4}{*}{5} & \multirow{4}{*}{5} & Коммутатор 3-го уровня & 1 & 8 \\ \cline{3-5}
            & & Коммутатор 2-го уровня & 1 & 24 \\ \cline{3-5}
            & & Патч-панель & 1 & 24 \\ \cline{3-5}
            & & ИБП & 1 & - \\ \hline
  \end{tabular}
  \caption{Сводная таблица по количеству требуемого коммутационного оборудования}
  \label{table:hw}
\end{table}

\subsubsection{Выбор  оборудования}
Для выбора оборудования (витая пара, патч-корд, информационных розеток, кабель канал, гофротруба, стойки, монтажные шкафы, коммутаторы, патч-панели, SFP-модули) выберем не менее трех производителей каждого из вида оборудования, и сделаем выбор на одном из них.

\begin{description}
\item[Витая пара]
  Кабель категории 5e - тип кабеля для передачи сигналов, состоящий из 4 витых пар. Используется в структурированных кабельных системах для компьютерных сетей, таких как Ethernet. Кабельный стандарт предоставляет производительность до 100 MHz и подходит для 10BASE-T, 100BASE-TX (Fast Ethernet), и 1000BASE-T (Gigabit Ethernet).
\\UTP – вид неэкранированного кабеля, т.е. кабель без дополнительной защиты от электромагнитных излучений. 
Так как схожи по характеристикам, выберем наиболее дешевый – кабель марки StabNet.
  \begin{table}[!htp]
    \centering
    \begin{tabular}{|l|l|l|l|l|l|}
      \hline
      Производитель & Тип кабеля & Категория & Длина & Количество жил & Стоимость \\ \hline
     StabNet & U/UTP & 5e & 305 & одножильный & 1560 рублей \\ \hline
      KRAULER & U/UTP & 5e & 305 & одножильный & 5360 рублей \\ \hline
      NIKOLAN & F/UTP & 5e & 305 & одножильный & 7600 рублей \\ \hline
    \end{tabular}
    \caption{Витая пара}
    \label{table:cabel}
  \end{table}
  
\item[Оптоволокно]
  Кабель марки Gembird является мультимодовым, в сравнении с кабелем марки LS-SC, и он дешевле (за 5 м). Следовательно, будем покупать его.
  \begin{table}[!htbp]
  \centering
    \begin{tabular}{|l|l|l|l|l|}%{|p{1cm}|p{1cm}|p{2cm}|p{2cm}|p{2cm}|}
      \hline
      Производитель & Тип кабеля & Длина & Разъемы & Стоимость \\ \hline
      Gembird & мультимодовый & 5 м & SC / CS & 451 рублей \\ \hline
      LS-SC & одномодовый & 1 м & SC / CS & 207 рублей \\ \hline
    \end{tabular}
    \caption{Оптоволокно}
    \label{table:optics}
  \end{table}
  
\item[Патч-корд]
  Патч-корды марки Netlan имееют оплётку ПВХ. При горении такой кабель образуют высокотоксичные галогенные соединения, очень опасные для человека. Поэтому купим патч-корд марки Nikomax, хотя он и стоит дороже.
\begin{table}[!htp]
    \centering
    \begin{tabular}{|l|l|l|l|l|l|l|l|}%{|p{1cm}|p{1cm}|p{2cm}|p{2cm}|p{2cm}|p{1cm}|p{1cm}|p{2cm}|}
      \hline
      Производитель & Экран & Тип & Категория & Длина & Оплётка & Штук & Стоимость \\ \hline
      Nikomax & UTP & прямой & 5e & 3 м & LSZH & 100 & 19200 рублей \\ \hline
      Netlan & UTP & прямой & 5e & 3 м & ПВХ & 100 & 6500 рублей \\ \hline

    \end{tabular}
    \caption{Патчкорд}
    \label{table:patchcord}
  \end{table}
  
\item[Информационные розетки]
  Так как мы используем и двухпортовые и однопортовые розетки, то выберем наиболее дешевые – марки Nikomax.
\begin{table}[!htp]
    \centering
    \begin{tabular}{|l|l|l|l|}%{|p{1cm}|p{1cm}|p{2cm}|p{2cm}|}
      \hline
      Производитель & Количество портов & Тип портов & Стоимость \\ \hline
      Nikomax & 2 & 8P8C & 280 рублей \\ \hline
      Netlan & 2 & 8P8C & 990 рублей \\ \hline
      Nikomax & 1 & 8P8C & 190 рублей \\ \hline
    \end{tabular}
    \caption{Информационные розетки}
    \label{table:infosocket}
  \end{table}
\item[Кабель-канал]
  Лучше выбрать кабель канал, которой имеет большую ширину. Выберем Ecoline.
\begin{table}[!htp]
    \centering
    \begin{tabular}{|l|l|l|l|l|}%{|p{1cm}|p{1cm}|p{2cm}|p{2cm}|p{2cm}|}
      \hline
      Производитель & Ширина & Высота & Длина & Стоимость \\ \hline
      Элекор & 20 мм & 10 мм & 2 м & 34 рублей \\ \hline
      Элекор & 100 мм & 60 мм & 2 м & 380 рублей \\ \hline
      Ecoline & 100 мм & 60 мм & 2 м & 261 рублей \\ \hline
    \end{tabular}
    \caption{Кабель-канал}
    \label{table:cabelcanal}
  \end{table}
\item[Гофротруба]
  Лучше выбрать гофротрубу с наибольшим диаметром, чтобы при добавлении новых кабелей, диаметра хватило для прохода кабеля. Выберем марку – Промрукав.
\begin{table}[!htp]
    \centering
    \begin{tabular}{|l|l|l|l|l|}%{|p{1cm}|p{1cm}|p{5cm}|p{2cm}|p{2cm}|}
      \hline
      Производитель & Диаметр & MAX t & MIN t & Стоимость \\ \hline
      Stout & 25 мм & 90 C & 40 C & 14 рублей \\ \hline
      THERMOTECH & 16 мм & 90 C & 40 C & 30 рублей \\ \hline
      Промрукав & 25 мм & 90 C & 40 C & 14 рублей \\ \hline
    \end{tabular}
    \caption{Гофротруба}
    \label{table:corrugatedtrube}
  \end{table}
\item[Стойка]
  Так как стойки имеют одинаковые параметры, выебер стойку марки Heperline, так как она более дешевая.
  \begin{table}[!htbp]
    \centering
    \begin{tabular}{|l|l|l|l|l|l|}%{|p{1cm}|p{1cm}|p{2cm}|p{2cm}|p{2cm}|p{2cm}|}
      \hline
      Производитель & Вид поставки & Вид монтажа & Конструктив & Стандарт упаковки & Стоимость \\ \hline
      TLK & разборный & напольный & 19 & 42U & 16 210 рублей \\ \hline
      Heperline & разборный & напольный & 19 & 42U & 6 990 рублей \\ \hline
    \end{tabular}
    \caption{Стойка}
    \label{table:rack}
  \end{table}
\item[Монтажный шкаф]
  Мы будем использовать монтажные шкафы настенные и напольные. Настенные будут использоваться на каждом этаже здания. Напольные будут использоваться только в серверной. Для настенного шкафа выберем марку Sysmatrix, а для напольного ЦМО (в серверной нужен шкаф с большой вместимостью)
  \begin{table}[!htp]
    \centering
    \begin{tabular}{|l|l|l|l|l|l|}%{|p{1cm}|p{1cm}|p{2cm}|p{2cm}|p{2cm}|p{2cm}|}
      \hline
      Производитель & Вид поставки & Вид монтажа & Конструктив & Стандарт упаковки & Стоимость \\ \hline
      Sysmatrix & разборный & настенный & 19 & 6U & 4 490 рублей \\ \hline
      ЦМО & разборный & настенный & 19 & 6U & 4 790 рублей \\ \hline
      TLK & разборный & настенный &19 &6U &5 670 рублей\\ \hline
      TLK & разборный & напольный & 19 & 18U & 18 890 рублей \\ \hline
      TLK & разборный & напольный & 19 & 24U & 22 190 рублей \\ \hline
      ЦМО & разборный & напольный & 19 & 42U & 40 070 рублей \\ \hline
    \end{tabular}
    \caption{Монтажный шкаф}
    \label{table:mounting}
  \end{table}
\item[Патч-панель]
  Так как разницы в пачт-панелях, кроме количества портов нет, выберем самые дешёвые, то есть марки Nikomax для 24 портов и в 5bites для 48 портов.
\begin{table}[!htp]
    \centering
    \begin{tabular}{|l|l|l|l|l|l|}%{|p{2cm}|p{1cm}|p{2cm}|p{2cm}|p{2cm}|p{2cm}|}
      \hline
      Производитель & Кол. портов & Категория & Способ крепления & Стандарт & Стоимость \\ \hline
      Nikomax & 24 & 5e & в стойку & UTP & 2 090 рублей \\ \hline
      Nikomax & 48 & 5e & в стойку & UTP &4 420 рублей \\ \hline
      NetLink & 24 & 5e & в стойку & UTP & 790 рублей \\ \hline
      5bites &48 & 5e & в стойку & UTP & 2 390 рублей \\ \hline
    \end{tabular}
    \caption{Патч-панель}
    \label{table:patchpanel}
  \end{table}
\item[SFP-модуль]
  Так как SFP модуль нужен для работы оптоволоконной сети, которая будет соединять здания, то лучше выбрать модуль с наибольшей скорость передачи данных, то есть марку SFP модуля – Форт-Телком.
\begin{table}[!htp]
    \centering
    \begin{tabular}{|l|l|l|l|l|l|}%{|p{2cm}|p{2cm}|p{2cm}|p{2cm}|p{2cm}|p{2cm}|}
      \hline
      Производитель & Скорость передачи данных & Дуплекс & Протоколы & Стандарт & Стоимость \\ \hline
      TP-LINK & до 1000 Мбит/с  & + & CSMA/CD, TCP/IP & 802.3z & 1590 рублей \\ \hline
      Форт-Телеком & 1.25Gb/s & + & CSMA/CD, TCP/IP & - &5 600 рублей \\ \hline
      TG-NET & до 1000 Мбит/с & + & CSMA/CD, TCP/IP & 802.3z & 1 284 рублей \\ \hline
    \end{tabular}
    \caption{SFP-модуль}
    \label{table:sfpmodule}
  \end{table}
\item[Коммутатор]
  Коммутаторы 2 уровня нужны на 24 портов. Лучше всего выбрать те, у которых пропускная способность выше: HP.
Коммутаторы 3 уровня будут использоваться для передачи информации между зданиями, поэтому необходима большая пропускная способность, следовательно, стоит выбрать коммутатор марки Cisco. Но данный коммутатор в 4 раза дороже коммутатора от HP, при этом пропускная способность больше всего в два раза. Поэтому, выберем коммутатор HP.
  \begin{table}[!htp]
    \centering
    \begin{tabular}{|l|l|l|l|l|l|l|}%{|p{2cm}|p{1cm}|p{2cm}|p{2cm}|p{2cm}|p{2cm}|p{2cm}|}
      \hline
      Производитель & Кол. портов & SFP порты & Управление & Скорость \newline Мбит/с & Пропускная способность &Стоимость \\ \hline
      Cisco & 24 & 2 & уровень 2 & 10/100/1000 & 8.8 Гбит/сек & 9 417 \\ \hline
      HP & 24 & 2 & уровень 2 & 10/100/1000 & 52 Гбит/сек 11 330 \\ \hline
      HP & 8 & 2 & уровень 3 & 10/100/1000 & 20 Гбит/сек & 24 242 \\ \hline
      HP & 8 & 2 &уровень 3 & 10/100/1000 &- & 51 873 \\ \hline
      Cisco & 8 & 4 & уровень 3 & 10/100/1000 & 46 Гбит/сек & 84 458 \\ \hline
      Cisco & 8 & 8 & уровень 3 & до 10000 & - &168 184 \\ \hline
    \end{tabular}
    \caption{Коммутатор}
    \label{table:commutator}
  \end{table}
\item[ИБП]
  Возьмем наиболее дешевый ИБП.
  \begin{table}[!htp]
    \centering
    \begin{tabular}{|l|l|l|l|}%{|p{2cm}|p{2cm}|p{2cm}|p{2cm}|}
      \hline
      Производитель & Мощность & Тип источника &Стоимость\\ \hline
      CyberPower & 360 Вт & резервный &  3 790 рублей \\ \hline
      Ippon & 360 Вт & резервный &4 640 рублей \\ \hline
      APC Pro & 230 Вт & интерактивный & 32 320 рублей \\ \hline
    \end{tabular}
    \caption{ИБП}
    \label{table:ibp}
  \end{table}
  
\item[Сервера]
  Был выбран универсальный сервер от Intel. На двух таких серверах методами виртуализации будут работать все реализованные сервисы ЛВС.
  \begin{table}[!htp]
    \centering
    \begin{tabular}{|l|l|}%{|p{3cm}|p{3cm}|}
      \hline
      Процессор & Intel Xeon E5-2630V4 \\ \hline
      Платформа & Intel R2308WTTYSR \\ \hline
      Контроллер диска & Adaptec ASR-8805 SGL \newline Adaptec AFM-700  \\ \hline
      Диски SAS & Seagate Cheetah 15K.7 3.6TB (6x600GB) \\ \hline
      Диски SSD & Intel 540S 240GB (2x120GB) \\ \hline
      ОЗУ & 64GB (4x16GB) \\ \hline
      Блок питания & 1100W \\ \hline
      Стоимость & 508196 рублей \\ \hline
    \end{tabular}
    \caption{Универсальный сервер Intel}
    \label{table:server}
  \end{table}
\end{description}

\section{Расчёт стоимости и эксплуатации локальной вычислительной сети}

\subsection{Расчёт стоимости оборудования и комплектующих}
% Расчёт стоимости оборудования и комплектующих
В таблице \ref{table:hwfinalcosts} сведены итоговые потребности в оборудовании и итоговая цена.

\begin{table}[!htbp]
  \centering
  \begin{tabular}{|p{3cm}|p{2cm}|p{2cm}|}
    \hline
    Наименование & Количество & Итоговая стоимость \\ \hline
    Коммутатор 2-го уровня 24 порта & 11 & 124630 \\ \hline
    Коммутатор 3-го уровня 8 портов & 5 & 121210 \\ \hline
    Витая пара & 1488 метров & 7800 \\ \hline
    Патч-корд & & \\ \hline
    Информационная розетка (однопортовая) & & \\ \hline
    Информационная розетка (двупортовая) & & \\ \hline
    Патч-панель 24 порта & 3 & 6270\\ \hline
    Патч-панель 48 портов & 4 & 9560\\ \hline
    Гофротруба & 300 метров & 4200 \\ \hline
    Кабель-канал & 1500 метров & 22500 \\ \hline
    Оптоволоконный кабель & 200 метров & 18040 \\ \hline
    SFP-модуль & 8 & 44800 \\ \hline
    Сервера & 2 & 1 016 392\\ \hline
    Монтажные шкафы (настенные) & 8 & 35920 \\ \hline
    Монтажные шкафы (напольные) & 1 & 40070 \\ \hline
    ИБП & 5 & 18950 \\ \hline
    Стойка серверная & 1 & 6990
    \textbf{Итого:} & & \\ \hline
  \end{tabular}
  \caption{Итоговая стоимость оборудования}
  \label{table:hwfinalcosts}
\end{table}

\subsection{Расчёт стоимости эксплуатации}
% Расчёт стоимости эксплуатации
В качестве операционной системы для серверов был выбран Red Hat Enterprise GNU/Linux Server. Сама операционная система распространяется на условиях лицензии GNU GPL v3, компания Red Hat обеспечивает техническую поддержку и сопровождение своим клиентам по телефону и сети Интернет.
Примерная стоимость - 180 тыс. рублей / год.

\subsection{Требования к составу обслуживающего персонала}
% Требования к составу обслуживающего персонала

\section*{Заключение}
\addcontentsline{toc}{section}{Заключение}
В процессе курсового проекта была спроектирована локальная сеть, построены поэтажные планы и планы корпусов. Так же были посчитаны длины кабелей, затраты на покупку и обслуживание оборудования, выдвинуты требования к обслуживающему персоналу.

\renewcommand\refname{Список использованных источников}
\nocite{*}
\bibliographystyle{./utf8gost705u}
\bibliography{biblio}
\end{document}