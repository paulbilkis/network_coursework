% Описание поэтажных планов зданий организации
\begin{description}
\item [1-е здание]
  Двухэтажное.
  \begin{description} 
  \item[1-й этаж]
  Отдел кадров (20 человек), отдел закупок (15 человек).
\item[2-й этаж]
  Отдел продаж (15 человек), бухгалтерия (20 человек).
\end{description}
Нумерация кабинетов слева направо по часовой стрелке, начиная с 1100 и заканчивая 1204. Вторая цифра обозначает этаж. В кабинетах с 1100 по 1103 размещаются сотрудники отдела кадров, в кабинетах с 1104 по 1106 размещаются сотрудники отдела закупок, в кабинетах с 1200 по 1202 размещаются сотрудники отдела продаж, в кабинетах с 1203 по 1206 располагается бухгалтерия. В каждом кабинете по 5 рабочих мест. Положение информационных розеток указано на плане. В помещении 1107 располагается стойка с сетевым оборудованием.
\item [2-е здание]
  Двухэтажное.
  \begin{description}
\item [1-й этаж]
  Отдел технического обеспечения (10 человек), хозяйственный отдел (15 человек).
\item[2-й этаж]
  IT-отдел (5 человек).
\end{description}
  Нумерация кабинетов слева направо по часовой стрелке, начиная с 2100 и заканчивая 2202. Вторая цифра обозначает этаж. В кабинетах с 2100 по 2101 размещаются сотрудники отдела технического обеспечения, в кабинетах с 2102 по 2103 размещаются сотрудники хозяйственного отдела, а в кабинетах с 2200 по 2201 размещаются IT-отдел.
  В кабинете 2102 10 рабочих мест, в остальных кабинетах по 5 рабочих мест. Положение информационных розеток указано на плане. В помещении 2202 располагается стойка с сетевым оборудованием.
\item [3-е здание]
  Двухэтажное.
 \begin{description} 
 \item[1-й этаж]
  Цех по пошиву женской верхней одежды (20 человек).
  \item[2-й этаж]
  Цех по пошиву лёгкого женского платья (15 человек).
\end{description}
  Оба цеха представляют собой помещения без деления на кабинеты. В каждом цеху есть потребнось в пяти компьютезированных рабочих местах, подключённых к  локальной вычислительной сети.
  
\item [4-е здание]
  Одноэтажное.\\
  В здании располагается цех по пошиву мужской верхней одежды - 10 человек.
  Цех представляет собой помещение без деления на кабинеты, занимает весь этаж. В цеху есть потребность в пяти компьютезированных рабочих местах, подключённых к локальной вычислительной сети.
\item [5-е здание]
  Одноэтажное.\\
  В здании располагается склад материалов и готовой продукции. Склад делится на два помещения, в одном располагается склад материалов, во втором - готовой продукции. В каждом помещении работает по 10 человек, имеется по 5 рабочих мест, подключённых к ЛВС.
\end{description}