% Выбор оборудования
\subsubsection{Количество коммутационного оборудования}
В каждом корпусе располагается монтажный шкаф или стойка, в котором установлено коммутационное оборудование.
Количество портов коммутатора и маршрутизатора зависит от числа компьютеров, которые необходимо подключить к сети.\\
Каждый компьютер соединяется с информационной розеткой посредством патч-корда, длиной 3 метра. Кабельная трасса от информационной розетки до патч-панели, расположенной в монтажном шкафу или стойке, проходит в кабель-канале. Соответствующие порты патч-панели и коммутатора соединяются друг с другом при помощи патч-корда, длиной 0,5 метра. Один порт в коммутаторе зарезервирован для соединения коммутатора с маршрутизатором. Маршрутизаторы соединяются между собой с помощью ВОЛС, поэтому предусмотрено наличие SFP-модуля для каждого маршрутизатора. Количество требуемое коммутационное и сопутствующее оборудование сведено в таблицу \ref{table:hw}

\begin{table}[h]
  \centering
  \begin{tabular}{|p{1cm}|p{1cm}|p{5cm}|p{2cm}|p{2cm}|}
    \hline
   № здания & № стойки & Название оборудования \newline или комплектующих & Количество & Количество \newline портов \\ \hline
    \multirow{4}{*}{1} & \multirow{4}{*}{1} & Коммутатор 3-го уровня & 1 & 8 \\ \cline{3-5}
            & & Коммутатор 2-го уровня & 4 & 24 \\ \cline{3-5}
            & & Патч-панель & 2 & 48 \\ \cline{3-5}
            & & ИБП & 1 & - \\ \hline
    \multirow{5}{*}{2} & \multirow{5}{*}{2} & Коммутатор 3-го уровня & 1 & 8 \\ \cline{3-5}
            & & Коммутатор 2-го уровня & 3 & 24 \\ \cline{3-5}
            & & Патч-панель & 1 &  48 \\ \cline{3-5}
            & & Патч-панель & 1 &  24 \\ \cline{3-5}
            & & ИБП & 1 & - \\ \hline
    \multirow{4}{*}{3} & \multirow{4}{*}{3} & Коммутатор 3-го уровня & 1 & 8 \\ \cline{3-5}
            & & Коммутатор 2-го уровня & 2 & 24 \\ \cline{3-5}
            & & Патч-панель & 1 & 48 \\ \cline{3-5}
            & & ИБП & 1 & - \\ \hline
    \multirow{4}{*}{4} & \multirow{4}{*}{4} & Коммутатор 3-го уровня & 1 & 8 \\ \cline{3-5}
            & & Коммутатор 2-го уровня & 1 & 24 \\ \cline{3-5}
            & & Патч-панель & 1 & 24 \\ \cline{3-5}
            & & ИБП & 1 & - \\ \hline
    \multirow{4}{*}{5} & \multirow{4}{*}{5} & Коммутатор 3-го уровня & 1 & 8 \\ \cline{3-5}
            & & Коммутатор 2-го уровня & 1 & 24 \\ \cline{3-5}
            & & Патч-панель & 1 & 24 \\ \cline{3-5}
            & & ИБП & 1 & - \\ \hline
  \end{tabular}
  \caption{Сводная таблица по количеству требуемого коммутационного оборудования}
  \label{table:hw}
\end{table}
\clearpage
\subsubsection{Выбор  оборудования}
Для выбора оборудования (витая пара, патч-корд, информационных розеток, кабель канал, гофротруба, стойки, монтажные шкафы, коммутаторы, патч-панели, SFP-модули) выберем не менее трех производителей каждого из вида оборудования, и сделаем выбор на одном из них.

\begin{description}
\item[Витая пара]
  Кабель категории 5e - тип кабеля для передачи сигналов, состоящий из 4 витых пар. Используется в структурированных кабельных системах для компьютерных сетей, таких как Ethernet. Кабельный стандарт предоставляет производительность до 100 MHz и подходит для 10BASE-T, 100BASE-TX (Fast Ethernet), и 1000BASE-T (Gigabit Ethernet).
\\UTP – вид неэкранированного кабеля, т.е. кабель без дополнительной защиты от электромагнитных излучений. 
Так как схожи по характеристикам, выберем наиболее дешевый – кабель марки StabNet.
  \begin{table}[!htp]
    \centering
    \begin{tabular}{|l|l|l|l|l|l|}
      \hline
      Производитель & Тип кабеля & Категория & Длина & Количество жил & Стоимость \\ \hline
     StabNet & U/UTP & 5e & 305 & одножильный & 1560 рублей \\ \hline
      KRAULER & U/UTP & 5e & 305 & одножильный & 5360 рублей \\ \hline
      NIKOLAN & F/UTP & 5e & 305 & одножильный & 7600 рублей \\ \hline
    \end{tabular}
    \caption{Витая пара}
    \label{table:cabel}
  \end{table}
  
\item[Оптоволокно]
  Кабель марки Gembird является мультимодовым, в сравнении с кабелем марки LS-SC, и он дешевле (за 5 м). Следовательно, будем покупать его.
  \begin{table}[!htbp]
  \centering
    \begin{tabular}{|l|l|l|l|l|}%{|p{1cm}|p{1cm}|p{2cm}|p{2cm}|p{2cm}|}
      \hline
      Производитель & Тип кабеля & Длина & Разъемы & Стоимость \\ \hline
      Gembird & мультимодовый & 5 м & SC / CS & 451 рублей \\ \hline
      LS-SC & одномодовый & 1 м & SC / CS & 207 рублей \\ \hline
    \end{tabular}
    \caption{Оптоволокно}
    \label{table:optics}
  \end{table}
  
\item[Патч-корд]
  Патч-корды марки Netlan имееют оплётку ПВХ. При горении такой кабель образуют высокотоксичные галогенные соединения, очень опасные для человека. Поэтому купим патч-корд марки Nikomax, хотя он и стоит дороже.
\begin{table}[!htp]
    \centering
    \begin{tabular}{|l|l|l|l|l|l|l|l|}%{|p{1cm}|p{1cm}|p{2cm}|p{2cm}|p{2cm}|p{1cm}|p{1cm}|p{2cm}|}
      \hline
      Производитель & Экран & Тип & Категория & Длина & Оплётка & Штук & Стоимость \\ \hline
      Nikomax & UTP & прямой & 5e & 0.5 м & LSZH & 300 & 19200 рублей \\ \hline
      Netlan & UTP & прямой & 5e & 0.5 м & ПВХ & 100 & 2 990 рублей \\ \hline
      Netlan & UTP & прямой & 5e & 0.5 м & ПВХ & 100 &3 790 рублей \\ \hline
    \end{tabular}
    \caption{Патчкорд}
    \label{table:patchcord}
  \end{table}
  
\item[Информационные розетки]
  Так как мы используем и двухпортовые и однопортовые розетки, то выберем наиболее дешевые – марки Nikomax.
\begin{table}[!htp]
    \centering
    \begin{tabular}{|l|l|l|l|}%{|p{1cm}|p{1cm}|p{2cm}|p{2cm}|}
      \hline
      Производитель & Количество портов & Тип портов & Стоимость \\ \hline
      Nikomax & 2 & 8P8C & 280 рублей \\ \hline
      Netlan & 2 & 8P8C & 990 рублей \\ \hline
      Nikomax & 1 & 8P8C & 190 рублей \\ \hline
    \end{tabular}
    \caption{Информационные розетки}
    \label{table:infosocket}
  \end{table}
\item[Кабель-канал]
  Лучше выбрать кабель канал, которой имеет большую ширину. Выберем Ecoline.
\begin{table}[!htp]
    \centering
    \begin{tabular}{|l|l|l|l|l|}%{|p{1cm}|p{1cm}|p{2cm}|p{2cm}|p{2cm}|}
      \hline
      Производитель & Ширина & Высота & Длина & Стоимость \\ \hline
      Элекор & 20 мм & 10 мм & 2 м & 34 рублей \\ \hline
      Элекор & 100 мм & 60 мм & 2 м & 380 рублей \\ \hline
      Ecoline & 100 мм & 60 мм & 2 м & 261 рублей \\ \hline
    \end{tabular}
    \caption{Кабель-канал}
    \label{table:cabelcanal}
  \end{table}
\item[Гофротруба]
  Лучше выбрать гофротрубу с наибольшим диаметром, чтобы при добавлении новых кабелей, диаметра хватило для прохода кабеля. Выберем марку – Промрукав.
\begin{table}[!htp]
    \centering
    \begin{tabular}{|l|l|l|l|l|}%{|p{1cm}|p{1cm}|p{5cm}|p{2cm}|p{2cm}|}
      \hline
      Производитель & Диаметр & MAX t & MIN t & Стоимость \\ \hline
      Stout & 25 мм & 90 C & 40 C & 14 рублей \\ \hline
      THERMOTECH & 16 мм & 90 C & 40 C & 30 рублей \\ \hline
      Промрукав & 25 мм & 90 C & 40 C & 14 рублей \\ \hline
    \end{tabular}
    \caption{Гофротруба}
    \label{table:corrugatedtrube}
  \end{table}
\item[Стойка]
  Так как стойки имеют одинаковые параметры, выебер стойку марки Heperline, так как она более дешевая.
  \begin{table}[!htbp]
    \centering
    \begin{tabular}{|l|l|l|l|l|l|}%{|p{1cm}|p{1cm}|p{2cm}|p{2cm}|p{2cm}|p{2cm}|}
      \hline
      Производитель & Вид поставки & Вид монтажа & Конструктив & Стандарт упаковки & Стоимость \\ \hline
      TLK & разборный & напольный & 19 & 42U & 16 210 рублей \\ \hline
      Heperline & разборный & напольный & 19 & 42U & 6 990 рублей \\ \hline
    \end{tabular}
    \caption{Стойка}
    \label{table:rack}
  \end{table}
\item[Монтажный шкаф]
  Мы будем использовать монтажные шкафы настенные и напольные. Настенные будут использоваться на каждом этаже здания. Напольные будут использоваться только в серверной. Для настенного шкафа выберем марку Sysmatrix, а для напольного ЦМО (в серверной нужен шкаф с большой вместимостью)
  \begin{table}[!htp]
    \centering
    \begin{tabular}{|l|l|l|l|l|l|}%{|p{1cm}|p{1cm}|p{2cm}|p{2cm}|p{2cm}|p{2cm}|}
      \hline
      Производитель & Вид поставки & Вид монтажа & Конструктив & Стандарт упаковки & Стоимость \\ \hline
      Sysmatrix & разборный & настенный & 19 & 6U & 4 490 рублей \\ \hline
      ЦМО & разборный & настенный & 19 & 6U & 4 790 рублей \\ \hline
      TLK & разборный & настенный &19 &6U &5 670 рублей\\ \hline
      TLK & разборный & напольный & 19 & 18U & 18 890 рублей \\ \hline
      TLK & разборный & напольный & 19 & 24U & 22 190 рублей \\ \hline
      ЦМО & разборный & напольный & 19 & 42U & 40 070 рублей \\ \hline
    \end{tabular}
    \caption{Монтажный шкаф}
    \label{table:mounting}
  \end{table}
\item[Патч-панель]
  Так как разницы в пачт-панелях, кроме количества портов нет, выберем самые дешёвые, то есть марки Nikomax для 24 портов и в 5bites для 48 портов.
\begin{table}[!htp]
    \centering
    \begin{tabular}{|l|l|l|l|l|l|}%{|p{2cm}|p{1cm}|p{2cm}|p{2cm}|p{2cm}|p{2cm}|}
      \hline
      Производитель & Кол. портов & Категория & Способ крепления & Стандарт & Стоимость \\ \hline
      Nikomax & 24 & 5e & в стойку & UTP & 2 090 рублей \\ \hline
      Nikomax & 48 & 5e & в стойку & UTP &4 420 рублей \\ \hline
      NetLink & 24 & 5e & в стойку & UTP & 790 рублей \\ \hline
      5bites &48 & 5e & в стойку & UTP & 2 390 рублей \\ \hline
    \end{tabular}
    \caption{Патч-панель}
    \label{table:patchpanel}
  \end{table}
\item[SFP-модуль]
  Так как SFP модуль нужен для работы оптоволоконной сети, которая будет соединять здания, то лучше выбрать модуль с наибольшей скорость передачи данных, то есть марку SFP модуля – Форт-Телком.
\begin{table}[!htp]
    \centering
    \begin{tabular}{|l|l|l|l|l|l|}%{|p{2cm}|p{2cm}|p{2cm}|p{2cm}|p{2cm}|p{2cm}|}
      \hline
      Производитель & Скорость передачи данных & Дуплекс & Протоколы & Стандарт & Стоимость \\ \hline
      TP-LINK & до 1000 Мбит/с  & + & CSMA/CD, TCP/IP & 802.3z & 1590 рублей \\ \hline
      Форт-Телеком & 1.25Gb/s & + & CSMA/CD, TCP/IP & - &5 600 рублей \\ \hline
      TG-NET & до 1000 Мбит/с & + & CSMA/CD, TCP/IP & 802.3z & 1 284 рублей \\ \hline
    \end{tabular}
    \caption{SFP-модуль}
    \label{table:sfpmodule}
  \end{table}
\item[Коммутатор]
  Коммутаторы 2 уровня нужны на 24 портов. Лучше всего выбрать те, у которых пропускная способность выше: HP.
Коммутаторы 3 уровня будут использоваться для передачи информации между зданиями, поэтому необходима большая пропускная способность, следовательно, стоит выбрать коммутатор марки Cisco. Но данный коммутатор в 4 раза дороже коммутатора от HP, при этом пропускная способность больше всего в два раза. Поэтому, выберем коммутатор HP.
  \begin{table}[!htp]
    \centering
    \begin{tabular}{|l|l|l|l|l|l|l|}%{|p{2cm}|p{1cm}|p{2cm}|p{2cm}|p{2cm}|p{2cm}|p{2cm}|}
      \hline
      Производитель & Кол. портов & SFP порты & Управление & Скорость \newline Мбит/с & Пропускная способность &Стоимость \\ \hline
      Cisco & 24 & 2 & уровень 2 & 10/100/1000 & 8.8 Гбит/сек & 9 417 \\ \hline
      HP & 24 & 2 & уровень 2 & 10/100/1000 & 52 Гбит/сек 11 330 \\ \hline
      HP & 8 & 2 & уровень 3 & 10/100/1000 & 20 Гбит/сек & 24 242 \\ \hline
      HP & 8 & 2 &уровень 3 & 10/100/1000 &- & 51 873 \\ \hline
      Cisco & 8 & 4 & уровень 3 & 10/100/1000 & 46 Гбит/сек & 84 458 \\ \hline
      Cisco & 8 & 8 & уровень 3 & до 10000 & - &168 184 \\ \hline
    \end{tabular}
    \caption{Коммутатор}
    \label{table:commutator}
  \end{table}
\item[ИБП]
  Возьмем наиболее дешевый ИБП.
  \begin{table}[!htp]
    \centering
    \begin{tabular}{|l|l|l|l|}%{|p{2cm}|p{2cm}|p{2cm}|p{2cm}|}
      \hline
      Производитель & Мощность & Тип источника &Стоимость\\ \hline
      CyberPower & 360 Вт & резервный &  3 790 рублей \\ \hline
      Ippon & 360 Вт & резервный &4 640 рублей \\ \hline
      APC Pro & 230 Вт & интерактивный & 32 320 рублей \\ \hline
    \end{tabular}
    \caption{ИБП}
    \label{table:ibp}
  \end{table}
  
\item[Сервера]
  Был выбран универсальный сервер от Intel. На двух таких серверах методами виртуализации будут работать все реализованные сервисы ЛВС.
  \begin{table}[!htp]
    \centering
    \begin{tabular}{|l|l|}%{|p{3cm}|p{3cm}|}
      \hline
      Процессор & Intel Xeon E5-2630V4 \\ \hline
      Платформа & Intel R2308WTTYSR \\ \hline
      Контроллер диска & Adaptec ASR-8805 SGL \newline Adaptec AFM-700  \\ \hline
      Диски SAS & Seagate Cheetah 15K.7 3.6TB (6x600GB) \\ \hline
      Диски SSD & Intel 540S 240GB (2x120GB) \\ \hline
      ОЗУ & 64GB (4x16GB) \\ \hline
      Блок питания & 1100W \\ \hline
      Стоимость & 508196 рублей \\ \hline
    \end{tabular}
    \caption{Универсальный сервер Intel}
    \label{table:server}
  \end{table}
\end{description}