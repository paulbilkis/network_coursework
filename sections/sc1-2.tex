% Структура организации и функции подразделений
Предприятие состоит из Административной части и Основного производства.

\subsubsection{Административная часть}
Административная часть состоит из следующих подразделений:
\begin{description}
\item [Отдел кадров]
  Отдел осуществляет контроль за текущей потребностью в кадрах, составляет списки резерва кадров, учувствует в работе по отбору кандидатов на работу, ведёт учет личного состава, оформляет документы о приеме сотрудников на работу, ведёт установленную документацию по кадрам.
\item [Отдел закупок]
  Отдел осуществляет поиск, анализ данных, выбор поставщиков необходимого оборудования. Проверяет поступающую продукцию.
\item [Отдел продаж]
  Отдел осуществляет рекламную деятельность, направленную на увеличение продаж продукции. Производит работу с клиентами, направленную на поддержание связей и увеличения объёмов взаимного сотрудничества.
\item [Отдел технического обеспечения]
  Отдел осуществляет ремонт или заменую сломанного оборудования.
\item [Бухгалтерия]
  Бухгалтерия занимается ведением бухгалтерского, налогового и управленческого учета финансово-хозяйственной деятельности завода, формирует бухгалтерскую, налоговую и управленческую отчетность, взаимодействует с государственными налоговыми органами, взаимодействует с финансовыми организациями в пределах своей компетенции, осуществляет платежи.
\item [IT-отдел]
  Отдел обеспечивает работоспособность локальной компьютерной сети, серверов и рабочих станций пользователей, обеспечивает информационную безопасность, антивирусную защиту, проводит техническое обслуживание и организацию ремонта вычислительной и оргтехники, обеспечивает рабочие станции и сервера свободным программным обеспечением.
\item [Хозяйственный отдел]
  Отдел обеспечивает хозяйственное, материально-техническое и социально-бытовое обслуживание издательства, контролирует исправность оборудования (лифтов, освещения, систем отопления и др.), оформляет документы, необходимые для заключения договоров на приобретение оборудования, обеспечивает подразделения канцелярскими принадлежностями, оборудованием, оргтехникой, мебелью, хозяйственными товарами, обеспечивает сохранность вышеуказанного оборудования, оформляет документы на техническое обслуживание и ремонт оргтехники и оборудования, занимается содержанием в надлежащем состоянии зданий и помещений предприятия.
\end{description}

\subsubsection{Основное производство}
\begin{description}
\item [Цех по пошиву лёгкого женского платья]
  Цех занимается пошивом лёгкого женского платья из готовых раскроек.
\item [Цех по пошиву женской верхней одежды]
  Цех занимается пошивом женской верхней одежды из готовых раскроек.
\item [Цех по пошиву мужской верхней одежды]
  Цех занимается пошивом мужской верхней одежды из готовых раскроек.
\item [Раскройный цех]
  Цех готовит раскройки для пошива изделий во всех пошивных цехах (кроме цеха по пошиву изделий из кожи).
\item [Склад]
  Склад готовой продукции.
\end{description}
