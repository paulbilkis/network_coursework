% Выбор топологии локальной вычислительной сети
Для формирования локальной вычислительной сети компьютеры могут соединяться различными способами. Конфигурация физического подключения компьютеров в сети называется сетевой топологией.
Виды сетевых топологий:
\begin{description}
\item[Полносвязная]
  Сеть, в которой каждый компьютер непосредственно связан со всеми остальными. Однако этот вариант громоздкий и неэффективный, потому что каждый компьютер в сети должен иметь большое количество коммуникационных портов, достаточное для связи с каждым из остальных компьютеров. Пример полносвязной топологии представлен на рисунке 1.


  Рисунок 1. Полносвязная топология.

\item[Шина]
  Топология данного типа представляет собой общий кабель (называемый шина или магистраль), к которому подсоединены все рабочие станции. На концах кабеля находятся терминаторы, для предотвращения отражения сигнала. Пример шинной топологии представлен на рисунке 2.

  Преимущества сетей шинной топологии:
  \begin{itemize}
  \item расход кабеля существенно уменьшен
  \item отказ одного из узлов не влияет на работу сети в целом;
  \item сеть легко настраивать и конфигурировать;
  \item сеть устойчива к неисправностям отдельных узлов.
  \end{itemize}
  Недостатки сетей шинной топологии:
  \begin{itemize}
  \item разрыв кабеля может повлиять на работу всей сети;
  \item ограниченная длина кабеля и количество рабочих станций;
  \item недостаточная надежность сети из-за проблем с разъемами кабеля;
  \item низкая производительность, обусловлена разделением канала между    всеми абонентами.
  \end{itemize}


  Рисунок 2. Шинная топология.

\item[Звезда]
  В сети, построенной по топологии типа «звезда», каждая рабочая станция подсоединяется кабелем (витой парой) к концентратору, или хабу. Концентратор обеспечивает параллельное соединение ПК и, таким образом, все компьютеры, подключенные к сети, могут общаться друг с другом.
  Данные от передающей станции сети передаются через хаб по всем линиям связи всем ПК. Информация поступает на все рабочие станции, но принимается только теми станциями, которым она предназначается. Так как передача сигналов в топологии физическая звезда является широковещательной, то есть сигналы от ПК распространяются одновременно во все направления, то логическая топология данной локальной сети является логической шиной. Пример звездной топологии представлен на рисунке 3.


  Преимущества сетей топологии звезда:
  \begin{itemize}
  \item легко подключить новый ПК;
  \item имеется возможность централизованного управления;
  \item сеть устойчива к неисправностям отдельных ПК и к разрывам соединения отдельных ПК.
  \end{itemize}

  Недостатки сетей топологии звезда:
  \begin{itemize}
  \item отказ хаба влияет на работу всей сети;
  \item большой расход кабеля.
  \end{itemize}
  
  Рисунок 3. Звездная топология.

\item[Кольцо]
  В сети с топологией типа «кольцо» все узлы соединены каналами связи в неразрывное кольцо (необязательно окружность), по которому передаются данные. Выход одного ПК соединяется со входом другого ПК. Начав движение из одной точки, данные, в конечном счете, попадают на его начало. Данные в кольце всегда движутся в одном и том же направлении.
  Принимающая рабочая станция распознает и получает только адресованное ей сообщение. В сети с топологией типа физическое кольцо используется маркерный доступ, который предоставляет станции право на использование кольца в определенном порядке. Логическая топология данной сети — логическое кольцо. Пример кольцевой топологии представлен на рисунке 4.

  Преимущества сетей топологии звезда:
  \begin{itemize}
  \item Данную сеть очень легко создавать и настраивать.
  \end{itemize}

  Недостатки сетей топологии звезда:
  \begin{itemize}
  \item К основному недостатку сетей топологии кольцо относится то, что повреждение линии связи в одном месте или отказ ПК приводит к неработоспособности всей сети.
  \end{itemize}

  Рисунок 4. Кольцевая топология.

\item[Ячеистая топология]
  Получается из полносвязной топологии путём удаления некоторых связей. Допускает соединения большого количества компьютеров и характерна для крупных сетей.

\item[Смешанная топология]
  Сетевая топология, преобладающая в крупных сетях с произвольными связями между компьютерами. В таких сетях можно выделить отдельные произвольно связанные фрагменты (подсети), имеющие типовую топологию, поэтому их называют сетями со смешанной топологией. Пример смешанной топологии представлен на рисунке 5.

  Рисунок 5. Смешанная топология.

  В курсовом проекте используется топология типа “звезда”, в которой компьютеры соединяются с центральной системой, называемой ядром или концентратором. В качестве ядра может использоваться не только концентратор, но и другое коммуникационное оборудование, например, маршрутизатор или коммутатор. Концентраторы имеют от 8 до 48 портов для подключения. В звездообразной топологии для соединения компьютеров с концентратором используется либо неэкранированный кабель типа витая пара 5-ой категории (UTP), либо экранированная пара (STP). Если расстояние (длина линии) от концентратора до каждого узла превышает 110 метров, необходимо использовать более дорогостоящую экранированную витую пару.
\end{description}