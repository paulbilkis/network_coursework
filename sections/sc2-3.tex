% Соединение удалённого коммуникационного оборудования
Для соединения коммутационного оборудования, которое находятся в различных зданиях, кабель витая пара чаще всего не используется, поэтому целесообразно рассмотреть две технологии: VPN и волокно-оптические системы связи.
VPN (Virtual Private Networks – Виртуальные частные сети) технология используется при объединении нескольких сетей, в одну виртуальную сеть, позволяя пользователям использовать ресурсы сетей всех офисов компании, как единую сеть. Отличное VPN сети от локальной сети состоит в том, что в VPN для передачи данных используется опорные сети оператора и Интернет каналы, а не кабель витой пары. Вся передаваемая по VPN информация надежно шифруется. 
Волокно-оптическая система связи (ВОЛС) – это вид системы передачи, при котором информация передается по оптическим диэлектрическим волноводам, известным под названием оптическое волокно. ВОЛС используется при построении объектов, в которых СКС должна объединить многоэтажное здание или при объединении территориально-разрозненных зданий. Оптически кабель может быть многомодовым (обеспечивает передачу сигналов на расстояние 1-5 км) и одномодовым (обеспечивает передачу сигналов на расстояние в десятки километров). Скорость передачи данных по оптоволокну ограничивается только пропускной способностью передающего и приемного модуля системы. Как правило, передача данных по оптоволокну составляет от 1 до 10 Гбит/с. Поэтому оптоволоконные системы чаще всего используется для передачи больших объемов информации, в том числе аудио и видеосигналов.
В рамках курсового проекта выбрана ВОЛС, поскольку её целесообразно использовать для соединения коммутационного оборудования, которое расположено в пределах километра друг от друга. Для прокладки ВОЛС применяется подземный метод, суть которого заключается в осуществлении монтажа кабеля по подземным коммуникациям. Данный метод позволяет оградить ВОЛС от нежелательных повреждений и негативного воздействия окружающей среды.
Для соединения двух удаленных маршрутизаторов друг с другом, необходимо наличие модулей SFP. Данные модули используются для присоединения платы маршрутизатора к ВОЛС.