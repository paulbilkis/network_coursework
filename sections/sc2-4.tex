% Выбор серверного оборудования
К коммутатору 3 уровня, который находится в главной серверной (монтажный шкаф 3) будут подключены следующие сервера:
\begin{description}
\item[Файловый сервер]
  Данный сервер представляет собой компьютер, первичной целью которого является обеспечение доступа к файлам (таких как документы, звуковые файлы, фотографии, изображения и т.д.), размещенных на его устройствах хранения информации, другим компьютерам издательства. Плюсами использования сервера являются: экономия пространства жесткого диска персональных компьютеров, обеспечение совместной работы пользователей с информационными ресурсами, надежность хранения информации. Сервер в локальной сети использует протокол SMB/CIFS (Windows и Unix-подобные операционные системы). Клиенты соединяются с сервером, используя протоколы TCP. После того, как соединение установлено, клиенты могут посылать команды серверу (эти команды называются SMB-команды), который дает им доступ к ресурсам, позволяет открывать, читать файлы, писать в файлы и, вообще, выполнять весь перечень действий, которые можно выполнять с файловой системой. В случае SMB, данные действия совершаются через сеть.

\item[Сервер печати]
  Он обеспечивает совместный доступ к принтеру, подключённому к серверу для определённого списка машин локальной сети.
\item[Сервер электронной почты]
  Сервер обеспечивает:
  \begin{itemize}
  \item Прием электронной почты завода из Интернет и ее раскладку по отдельным ящикам внутри общего домена
  \item Проверку входящей почты на наличие вирусов и их обезвреживание в случае необходимости
  \end{itemize}
  Размеры почтовых ящиков пользователей ограничены только возможностями сервера. На сервере организован электронный документооборот и обмен сообщениями внутри издательства. Вся конфиденциальная информация остается в пределах локальной сети.
  Когда пользователь набрал сообщение и посылает его получателю, почтовый клиент взаимодействует с почтовым сервером, используя протокол SMTP. Почтовый сервер отправителя взаимодействует с почтовым сервером получателя. На почтовом сервере получателя сообщение попадает в почтовый ящик, откуда при помощи агента доставки сообщений (MDA) доставляется клиенту получателя. Для финальной доставки полученных сообщений используется протокол POP3 или IMAP.
\item[Сервер баз данных]
  Сервер БД обслуживает базу данных и отвечает за целостность и сохранность данных, а также обеспечивает операции ввода-вывода при доступе к информации. Применяемая база данных – MySQL. В базе данных содержится полная информация о перечне часов, выпускаемых заводом.
\item[Веб-сервер]
  Данный сервер обеспечивает размещение и выдачу по запросу любой информации в формате HTML, что позволяет издательству организовать собственное Интернет представительство, прорекламировать свои услуги и продукты, довести до конечных пользователей прейскурант, получить обратную связь.
  Клиент, которым обычно является веб-браузер, передаёт веб-серверу запросы на получение ресурсов, обозначенных URL-адресами. Ресурсы – это HTML-страницы, изображения, файлы, медиа-потоки или другие данные, которые необходимы клиенту. В ответ веб-сервер передаёт клиенту запрошенные данные. Такой обмен происходит по протоколу HTTP.
\item[Сервер резервного копирования файлов]
  Сервер производит копирование наиболее ценной информации на магнитную ленту или магнитооптические диски. Если какой-либо из серверов полностью вышел из строя или вся база данных была случайно удалена, сервер резервного копирования становиться незаменимым. Сервер резервного копирования может некорректно сохранить некоторые данные (например, сообщения с почтового сервера). Чтобы этого не происходило, существуют программы-агенты для каждого из сетевых приложений. Они поставляются как дополнение к серверу резервного копирования.
\item[Прокси-сервер]
  Данный сервер действует как посредник, помогая пользователям получить информацию из Интернета (трансляция запросов к WWW и FTP ресурсам) и при этом, обеспечивая защиту сети, может сохранять часто запрашиваемую информацию в кэш-памяти на локальном диске, быстро доставляя её пользователям без повторного обращения к Интернету, делая потребление Интернет трафика более экономичным. Кроме того, он может вводить ограничение доступа к ресурсам Интернет для рабочих станций локальной сети в соответствии с внутрикорпоративной политикой безопасности.
  Прокси-сервер сконфигурирован таким образом, что принимает или отвергает определённые типы сетевых запросов, поступающие как из локальной сети, так и из Интернета. В такой конфигурации прокси-сервер становиться межсетевым экраном – брандмауэром. Он обеспечивает:
  \begin{itemize}
  \item Защиту локальной сети предприятия от несанкционированного и неправомерного доступа из сети Интернет к ресурсам локальной сети организации;
  \item Трансляцию внутренних адресов локальной сети в публичные Интернет адреса (NAT – network address translation);
  \item Ограничение доступа в Интернет машин локальной сети (Proxy, ACL);
  \item Ограничение ресурсов сети Интернет для доступа из локальной сети (возможно разграничение доступа к серверам WWW в зависимости от содержания).
  \end{itemize}
\end{description}