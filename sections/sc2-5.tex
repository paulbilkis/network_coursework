% Адресация

Протокол IP (Internet Protocol) входит в состав стека протоколов TCP/IP и является основным протоколом сетевого уровня, использующимся в Интернет и обеспечивающим единую схему логической адресации устройств в сети и маршрутизацию данных. Подтверждение получения пакетов и повторное обращение за потерянными пакетами входят в круг обязанностей протокола более высокого уровня, например TCP.
С точки зрения протокола IP, сеть рассматривается как логическая совокупность взаимосвязанных объектов, каждый из которых представлен уникальным IP-адресом, называемых узлами (IP-узлами) или хостами (host).
IP-адрес – это уникальный числовой адрес, однозначно идентифицирующий узел, группу узлов или сеть. IP-адрес имеет длину 4 байта и обычно записывается в виде четырех чисел (так называемых “октетов”), разделенных точками – W.X.Y.Z , каждое из которых может принимать значения в диапазоне от 0 до 255.
Существует 5 классов IP-адресов – A, B, C, D, E. Принадлежность IP-адреса к тому или иному классу определяется значением первого октета (W). В таблице \ref{table:ipexample} показано соответствие значений первого октета и классов адресов.

\begin{table}[h]
  \centering
  \begin{tabular}{l|l|l|l|l|l}
    Класс IP-адресса & A & B & C & D & E \\ \hline
    Диапазон первого октета & 1-127 & 128-191  & 192-223 &224-239  & 240-247 \\ \hline
  \end{tabular}
  \caption{Диапазон первого октета для различных классов IP-адресов}
  \label{table:ipexample}
\end{table}

Частное адресное пространство определяется следующими адресными блоками:
\begin{itemize}
\item От 10.0.0.1 до 10.255.244.254
\item От 172.16.0.1 до 172.31.255.254
\item От 192.168.0.1 до 192.168.255.254
\end{itemize}
Данные адреса используются в локальных сетях небольших организаций и не требуют регистрации.
Если количество компьютеров в сети не будет превышать 254, то рекомендуется использовать адреса из диапазона от 192.168.0.1 до 192.168.0.254 с маской подсети 255.255.255.0. Тогда 192.168.0.0 будет номер сети, а адреса компьютеров от 1 до 254. Если компьютеров будет больше, чем 254, то можно использовать диапазон от 172.16.0.1 до 172.31.255.254 с маской подсети 255.255.0.0. Тогда 172.16.0.0 будет номером сети, а адреса компьютеров от 0.1 до 255.254 (это более 65000 адресов).
Присваивается IP-адрес компьютеру либо вручную (статический адрес), либо компьютер получает его автоматически с сервера (динамический адрес). Статический адрес прописывается администратором сети в настройках протокола TCP/IP на каждом компьютере сети и жестко закрепляется за компьютером. В курсовом проекте используется статическая адресация.
IP-маршрутизация – процесс выбора пути для передачи пакета в сети. Под путем (маршрутом) понимается последовательность маршрутизаторов, через которые проходит пакет по пути к узлу-назначению. IP-маршрутизатор – это специальное устройство, предназначенное для объединения сетей и обеспечивающее определение пути прохождения пакетов в составной сети. Маршрутизатор должен иметь несколько IP-адресов с номера сетей, соответствующими номерам объединяемых сетей. Маршрутизация осуществляется на узле-отправителе в момент отправки IP-пакета, а затем на IP-маршрутизаторах.
Когда требуется отправить пакет узлу с определенным IР-адресом, то узел-отправитель выделяет с помощью маски подсети из собственного IР-адреса и IР—адреса получателя номера сетей. Далее номера сетей сравниваются и если они совпадают, то пакет направляется непосредственно получателю, в противном случае - маршрутизатору, чей адрес указан в настройках протокола IP. Если ни узле не настроен адрес маршрутизатора, то доставка данных получателю, расположенному в другой сети, окажется невозможной.
Выбор пути на маршрутизаторе осуществляется на основе информации, представленной в таблице маршрутизации. Таблица маршрутизации – это специальная таблица, сопоставляющая IP-адресам сетей адреса следующих маршрутизаторов, на которые следует отправлять пакеты с целью их доставки в эти сети. Обязательной записью в таблице маршрутизации является так называемый маршрут по умолчанию, содержащий информацию о том, как направлять пакеты в сети, адреса которых не присутствуют в таблице, поэтому нет необходимости описывать в таблице маршруты для всех сетей. Таблицы маршрутизации могут строится «вручную» администратором или динамически с помощью специальных протоколов.

В таблице \ref{table:ipnumtable} приведены сведения о требуемом числе IP-адресов для каждого подразделения производства.



\begin{table}[!htbp]
  \centering
  \begin{tabular}{p{2cm}|p{5cm}|p{2cm}|p{2cm}|p{2cm}|p{2cm}|p{2cm}}
    № здания & Название \newline подразделения & Число \newline рабочих мест & Число IP-адресов & Число IP-адресов в подсети & Число IP-адресов в резерве & Количество портов коммутатора \\ \hline
    \multirow{4}{*}{1} & Отдел кадров & 20+2 & 25 & 32 & 7 & 24  \\ 
             & Отдел закупок & 15+2 & 20 & 32 & 12  & 24 \\ 
             & Отдел продаж & 15+2 & 20 & 32 &  12 & 24 \\ 
             & Бухгалтерия & 20+2 & 25 & 32 & 7  & 24 \\ \hline
    \multirow{3}{*}{2} & Отдел технического обеспечения & 10+1 & 14 & 16 & 2 & 24   \\ 
             & Хозяйственный отдел & 15+2 & 20 & 32 & 12 & 24   \\ 
             & IT-отдел & 5+1 & 9 & 16 & 7 & 24  \\ \hline
    \multirow{2}{*}{3} & Цех по пошиву женской верхней одежды & 5+2 & 10 & 16 & 6 & 24 \\ 
             & Цех по пошиву лёгкого женского платья & 5+2 & 10 & 16 & 6 & 24 \\ \hline
    4 & Цех по пошиву мужской верхней одежды & 5+2 & 10 & 16 & 6 & 24 \\ \hline
    5 & Склад & 10+4 & 17 & 32 & 15 & 24 \\  \hline
    \textbf{Итого} & & 125+22 & 180 & 272 & 92 & 24 (11 штук)\\ \hline
  \end{tabular}
  \caption{Сводная таблица по требуемому количеству IP-адресов}
  \label{table:ipnumtable}
\end{table}


В ЛВС находятся 125 компьютеров. Согласно теоретическим сведениям, если количество компьютеров не превышает 254, то рекомендуется использовать адреса из диапазона от 192.168.0.1 до 192.168.0.254 с маской подсети 255.255.255.0. Тогда 192.168.0.0 будет номер сети, а адреса компьютеров от 1 до 254.
Исходя из таблицы \ref{table:ipnumtable} необходимо 6 подсетей на 32 адреса и 5 подсетей на 16 адресов. Адреса сетей и диапазоны адресов узлов представлены в таблице \ref{table:ipfinal}

\begin{table}[!htbp]
  \centering
  \begin{tabular}{|l|l|l|l|}
    Адрес подсети & Диапазон адресов узлов подсети & Адрес широковещательной  рассылки & Маска подсети \\ \hline
    \multicolumn{4}{c}{6 подсетей по 32 адреса} \\ \hline
    192.168.0.0 & 192.168.0.1-192.168.0.30 & 192.168.0.31 & \multirow{6}{*}{255.255.255.224} \\ \cline{1-3}
    192.168.0.32 & 192.168.0.33-192.168.0.62 & 192.168.0.63  &  \\ \cline{1-3}
    192.168.0.64 & 192.168.0.65-192.168.0.94 & 192.168.0.95  &  \\ \cline{1-3}
    192.168.0.96 & 192.168.0.97-192.168.0.126 & 192.168.0.127 &  \\ \cline{1-3}
    192.168.0.128 & 192.168.0.129-192.168.0.158 & 192.168.0.159 &  \\ \cline{1-3}
    192.168.0.160 & 192.168.0.161-192.168.0.190 & 192.168.0.191 &  \\ \hline
    \multicolumn{4}{c}{5 подсетей по 16 адресов} \\ \hline
    192.168.1.0 & 192.168.1.1-192.168.1.14 & 192.168.1.15 & \multirow{5}{*}{255.255.255.240} \\ \cline{1-3}
    192.168.1.16 & 192.168.1.17-192.168.1.30 & 192.168.1.31 &  \\ \cline{1-3}
    192.168.1.48 & 192.168.1.49-192.168.1.62 & 192.168.1.63 &  \\ \cline{1-3}
    192.168.1.64 & 192.168.1.65-192.168.1.78 & 192.168.1.79 &  \\ \cline{1-3}
    192.168.1.80 & 192.168.1.81-192.168.1.94 & 192.168.1.95 &  \\ \cline{1-3}
    
  \end{tabular}
  \caption{Распределение IP адресов}
  \label{table:ipfinal}
\end{table}

